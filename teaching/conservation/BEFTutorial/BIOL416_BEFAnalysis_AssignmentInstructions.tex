\documentclass[]{article}
\usepackage{lmodern}
\usepackage{amssymb,amsmath}
\usepackage{ifxetex,ifluatex}
\usepackage{fixltx2e} % provides \textsubscript
\ifnum 0\ifxetex 1\fi\ifluatex 1\fi=0 % if pdftex
  \usepackage[T1]{fontenc}
  \usepackage[utf8]{inputenc}
\else % if luatex or xelatex
  \ifxetex
    \usepackage{mathspec}
  \else
    \usepackage{fontspec}
  \fi
  \defaultfontfeatures{Ligatures=TeX,Scale=MatchLowercase}
\fi
% use upquote if available, for straight quotes in verbatim environments
\IfFileExists{upquote.sty}{\usepackage{upquote}}{}
% use microtype if available
\IfFileExists{microtype.sty}{%
\usepackage{microtype}
\UseMicrotypeSet[protrusion]{basicmath} % disable protrusion for tt fonts
}{}
\usepackage[margin=1in]{geometry}
\usepackage{hyperref}
\hypersetup{unicode=true,
            pdftitle={R Assignment 1: Biodiversity - Ecosystem Function},
            pdfborder={0 0 0},
            breaklinks=true}
\urlstyle{same}  % don't use monospace font for urls
\usepackage{graphicx,grffile}
\makeatletter
\def\maxwidth{\ifdim\Gin@nat@width>\linewidth\linewidth\else\Gin@nat@width\fi}
\def\maxheight{\ifdim\Gin@nat@height>\textheight\textheight\else\Gin@nat@height\fi}
\makeatother
% Scale images if necessary, so that they will not overflow the page
% margins by default, and it is still possible to overwrite the defaults
% using explicit options in \includegraphics[width, height, ...]{}
\setkeys{Gin}{width=\maxwidth,height=\maxheight,keepaspectratio}
\IfFileExists{parskip.sty}{%
\usepackage{parskip}
}{% else
\setlength{\parindent}{0pt}
\setlength{\parskip}{6pt plus 2pt minus 1pt}
}
\setlength{\emergencystretch}{3em}  % prevent overfull lines
\providecommand{\tightlist}{%
  \setlength{\itemsep}{0pt}\setlength{\parskip}{0pt}}
\setcounter{secnumdepth}{0}
% Redefines (sub)paragraphs to behave more like sections
\ifx\paragraph\undefined\else
\let\oldparagraph\paragraph
\renewcommand{\paragraph}[1]{\oldparagraph{#1}\mbox{}}
\fi
\ifx\subparagraph\undefined\else
\let\oldsubparagraph\subparagraph
\renewcommand{\subparagraph}[1]{\oldsubparagraph{#1}\mbox{}}
\fi

%%% Use protect on footnotes to avoid problems with footnotes in titles
\let\rmarkdownfootnote\footnote%
\def\footnote{\protect\rmarkdownfootnote}

%%% Change title format to be more compact
\usepackage{titling}

% Create subtitle command for use in maketitle
\newcommand{\subtitle}[1]{
  \posttitle{
    \begin{center}\large#1\end{center}
    }
}

\setlength{\droptitle}{-2em}

  \title{R Assignment 1: Biodiversity - Ecosystem Function}
    \pretitle{\vspace{\droptitle}\centering\huge}
  \posttitle{\par}
    \author{}
    \preauthor{}\postauthor{}
    \date{}
    \predate{}\postdate{}
  

\begin{document}
\maketitle

In Tutorials 5 and 6, we have been analysing different facets of
biodiversity and how it relates to ecosystem function in experimental
plant communities.

The four measures of biodiversity we calculated are:

\begin{enumerate}
\def\labelenumi{\arabic{enumi}.}
\tightlist
\item
  Species richness
\item
  Ln-transformed species richness
\item
  Phylogenetic diversity using a phylogeny based on on genetic
  substitutions (the original \texttt{phylogeny.phy} file from Tutorial
  6)
\item
  Phylogenetic diversity with a phylogenetic tree that has been
  transformed to have branch lengths that reflect time rather than
  substitutions (the phylogeny created by the \texttt{chronoMPL()}
  function from the \texttt{ape} library)
\end{enumerate}

Now that you have examined the biodiversity - ecosystem function
relationship in these plant communities, please write up a written
report that addresses the following:

\begin{enumerate}
\def\labelenumi{\arabic{enumi}.}
\item
  \emph{2 points}: Show your plots of species richness against
  above-ground biomass - these plots should include axes titles with
  appropriate units of measure and with an overlaid line of best fit
  from the linear model. Show both untransformed and transformed species
  richness' relationships.
\item
  \emph{1 point}: Describe the shape of the relationship between species
  richness and productivity. Also describe the linear model that
  generates this observed relationship. How does the observed
  relationship compare to the expected shape of the B-EF relationship?
  What mechanisms of the B-EF relationship does your model predict? How
  confident are you of this?
\item
  \emph{1 point}: Discuss the likely ecological process(es) that you
  think shaped the observed richness-biomass relationship. Consider the
  alternative processes and how you might test your hypothesis for which
  one caused the B-EF relationship.
\item
  \emph{2 points}: Like Question 1, show the plots of PD versus
  productivity for the substitution-based PD and the transformed
  phylogeny's PD. Again, axes should be labelled with appropriate units
  with an overlaid line of best fit from the linear model.
\item
  \emph{1 point}: Is the PD model you generated a better fit than the
  models with species richness? What did you initially predict would be
  a better fitting model? Did the model support your predictions?
\item
  \emph{1 point}: How did transforming the branch lengths in the
  phylogenetic tree alter your model fitting results? Discuss why this
  phylogeny was better or worse in explaining the observed variation in
  above-ground biomass.
\end{enumerate}

This assignment is worth 10\% of your final grade. 2 points of the 10
total will come from your R code's ability to execute and produce the
necessary plots.

Your report should be between 500-800 words long (approximately 2
type-written pages). Submit your written answers to the questions above
AND your R code to Canvas in your section's \emph{Tutorial 5/6 -
Biodiversity \& Ecosystem Function} folder.

Your code and written answers should be submitted \textbf{2pm on Monday,
February 25 or Wednesday, February 28} depending on whether you are in
the Monday or Wednesday tutorial section.


\end{document}
