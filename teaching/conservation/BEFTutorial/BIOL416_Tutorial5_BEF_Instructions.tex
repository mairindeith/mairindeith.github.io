\documentclass[]{article}
\usepackage{lmodern}
\usepackage{amssymb,amsmath}
\usepackage{ifxetex,ifluatex}
\usepackage{fixltx2e} % provides \textsubscript
\ifnum 0\ifxetex 1\fi\ifluatex 1\fi=0 % if pdftex
  \usepackage[T1]{fontenc}
  \usepackage[utf8]{inputenc}
\else % if luatex or xelatex
  \ifxetex
    \usepackage{mathspec}
  \else
    \usepackage{fontspec}
  \fi
  \defaultfontfeatures{Ligatures=TeX,Scale=MatchLowercase}
\fi
% use upquote if available, for straight quotes in verbatim environments
\IfFileExists{upquote.sty}{\usepackage{upquote}}{}
% use microtype if available
\IfFileExists{microtype.sty}{%
\usepackage{microtype}
\UseMicrotypeSet[protrusion]{basicmath} % disable protrusion for tt fonts
}{}
\usepackage[margin=1in]{geometry}
\usepackage{hyperref}
\hypersetup{unicode=true,
            pdftitle={BIOL 416 R Tutorial 5 - Biodiversity-ecosystem function in plant communities},
            pdfauthor={Mairin Deith},
            pdfborder={0 0 0},
            breaklinks=true}
\urlstyle{same}  % don't use monospace font for urls
\usepackage{graphicx,grffile}
\makeatletter
\def\maxwidth{\ifdim\Gin@nat@width>\linewidth\linewidth\else\Gin@nat@width\fi}
\def\maxheight{\ifdim\Gin@nat@height>\textheight\textheight\else\Gin@nat@height\fi}
\makeatother
% Scale images if necessary, so that they will not overflow the page
% margins by default, and it is still possible to overwrite the defaults
% using explicit options in \includegraphics[width, height, ...]{}
\setkeys{Gin}{width=\maxwidth,height=\maxheight,keepaspectratio}
\IfFileExists{parskip.sty}{%
\usepackage{parskip}
}{% else
\setlength{\parindent}{0pt}
\setlength{\parskip}{6pt plus 2pt minus 1pt}
}
\setlength{\emergencystretch}{3em}  % prevent overfull lines
\providecommand{\tightlist}{%
  \setlength{\itemsep}{0pt}\setlength{\parskip}{0pt}}
\setcounter{secnumdepth}{0}
% Redefines (sub)paragraphs to behave more like sections
\ifx\paragraph\undefined\else
\let\oldparagraph\paragraph
\renewcommand{\paragraph}[1]{\oldparagraph{#1}\mbox{}}
\fi
\ifx\subparagraph\undefined\else
\let\oldsubparagraph\subparagraph
\renewcommand{\subparagraph}[1]{\oldsubparagraph{#1}\mbox{}}
\fi

%%% Use protect on footnotes to avoid problems with footnotes in titles
\let\rmarkdownfootnote\footnote%
\def\footnote{\protect\rmarkdownfootnote}

%%% Change title format to be more compact
\usepackage{titling}

% Create subtitle command for use in maketitle
\newcommand{\subtitle}[1]{
  \posttitle{
    \begin{center}\large#1\end{center}
    }
}

\setlength{\droptitle}{-2em}

  \title{BIOL 416 R Tutorial 5 - Biodiversity-ecosystem function in plant
communities}
    \pretitle{\vspace{\droptitle}\centering\huge}
  \posttitle{\par}
    \author{Mairin Deith}
    \preauthor{\centering\large\emph}
  \postauthor{\par}
    \date{}
    \predate{}\postdate{}
  

\begin{document}
\maketitle

{
\setcounter{tocdepth}{2}
\tableofcontents
}
The purpose of this tutorial is to investigate how the \textbf{species
richness} of plant communities influences the \textbf{biomass produced}
by those communities.

We will be working a dataset that includes observations on the species
richness and above-ground biomass production of study plots of plant
communities. The data used in this analysis is contained in the csv file
\textbf{BIOL416\_Tutorial5\_BEFData.csv}, which can be downloaded off of
Canvas in the Tutorial 5 folder
\href{https://canvas.ubc.ca/users/198718/files/3931377}{or at this
link.}

Try to follow these steps on your own first, but don't worry if you get
stuck. We will go through the analysis together as a class,
step-by-step, after you have a chance to try each stage of analysis.

\subsection{Instructions}\label{instructions}

\subsubsection{Load in the BEF data and review its
structure}\label{load-in-the-bef-data-and-review-its-structure}

\begin{enumerate}
\def\labelenumi{\arabic{enumi}.}
\item
  Start up a new R script and load the .csv file into your R session -
  you can do this using the \texttt{file.choose()} function we have used
  previously, or by using file paths in your R script.
  \href{http://rfunction.com/archives/1001}{Click here for more
  information on using file paths in R.}
\item
  Examine the BEF data. What data are represented in columns? In rows?
  What data populates the data frame?
\end{enumerate}

\subsubsection{BEF analysis: overview}\label{bef-analysis-overview}

We want to find out if there is a significant relationship between
species richness (SR) and biomass production (B) in these plant
communities. To test for statistical significant, we will conduct two
linear regressions between SR and B: one in which SR is transformed by
the natural logarithm, ln, and another where SR is in its raw state.

To do this, we will:

\begin{enumerate}
\def\labelenumi{\alph{enumi})}
\item
  Calculate richness and mean biomass in each community
\item
  Run a linear regression of both: \[Biomass \sim m * Richness + b\]
  \[Biomass \sim m * ln(Richness) + b\] \ldots{}where \(m\) is the
  coefficient/slope for \(Richness\) and \(b\) is the y-intercept of the
  linear regression.
\item
  Interpret and plot the results of (b)
\end{enumerate}

\textbf{Question: Why are we interested in comparing ln-transformed and
un-transformed species richness data?}

\subsubsection{BEF analysis:
Step-by-step}\label{bef-analysis-step-by-step}

\begin{enumerate}
\def\labelenumi{\arabic{enumi}.}
\tightlist
\item
  Start with just one plot, \texttt{X2}. Create a data frame that
  includes \textbf{only observations for plot \texttt{X2}}.
\end{enumerate}

\emph{Hint: take a look at the \texttt{subset()} function if you're
stuck}.

\begin{enumerate}
\def\labelenumi{\arabic{enumi}.}
\setcounter{enumi}{1}
\item
  Calculate the species richness in the \texttt{X2} plot. SR changes
  year-per-year in some plots; \textbf{only calculate SR for the first
  year of observations!}
\item
  Calculate the average biomass production (averaged across all years of
  observation) in \texttt{X2}.
\item
  Repeat steps 1-3 for the remaining plots.
\end{enumerate}

\emph{Hint: You \emph{should not} do this by hand! we can use a
\texttt{for()} loop for this. If you don't know what a for-loop is, stop
here - we will review this as a class. If you're feeling ambitious,
\href{https://www.r-bloggers.com/how-to-write-the-first-for-loop-in-r/}{you
can read about how to write a for-loop here}}.

\begin{enumerate}
\def\labelenumi{\arabic{enumi}.}
\setcounter{enumi}{4}
\item
  Don't forget to store your results somewhere! Make a data frame of
  your BEF values including the Plot ID, species richness, and biomass
  production for each plot. It is easiest to do this before you run the
  loop in Step 4.
\item
  Create and run a linear model that looks at:

  \begin{enumerate}
  \def\labelenumii{\alph{enumii}.}
  \item
    Biomass as a function of species richness
    \[Biomass \sim m * Richness + b\]
  \item
    Biomass as a function of ln-transformed species richness.
    \emph{Hint: in R, \texttt{log(X)} takes the natural logarithm of
    some object, X. \texttt{log10()} is used to take the base-10
    logarithm that you're probably used to}.
    \[Biomass \sim m * ln(Richness) + b\]
  \end{enumerate}
\item
  Plot each of these relationships with a scatterplot and overlay the
  linear relationships created in Step 6 on the data.
\item
  Interpret these relationships - what do the coefficients and
  intercepts tell us about species richness and how it influences
  biomass production?
\item
  Compare the ability of the ln-transformed and untransformed species
  richness to predict biomass production.
\end{enumerate}

\emph{Hint: you can do this by considering the
\texttt{Multiple\ R-squared} result from the linear model objects
created in Step 6; we will discuss what R-squared is in class if you are
unfamiliar}.

\begin{enumerate}
\def\labelenumi{\arabic{enumi}.}
\setcounter{enumi}{9}
\tightlist
\item
  Write up your observations to answer the questions posted on Canvas in
  the Tutorial 5 - Biodiversity \& Ecosystem Function folder.
\end{enumerate}


\end{document}
